\documentclass[italian, a4paper, flagCMYK]{europecv}
\usepackage{graphicx}
\usepackage[a4paper, left=2.5cm, right=2cm, top=2cm, bottom=2cm]{geometry}
\usepackage{hyperref}

\begin{document}
%
% Personal data
%
\ecvname{Scaramuzza Paolo}
\ecvaddress{via Androna della Pergola 14\\& 34170 Gorizia (GO), Italia}
\ecvtelephone{346-1724250}
\ecvemail{paolo.scaramuzza@studenti.unipd.it}
\ecvdateofbirth{16 Dicembre, 1992}
\ecvnationality{Italiana}
\ecvbeforepicture{\raggedleft\ecvspace{-3cm}}
\ecvpicture[height=3.5cm]{img/photo.jpg}
%
% Curriculum
%
\begin{europecv}
\ecvpersonalinfo

\ecvsection{Istruzione e formazione}
\ecvitem{dal 2009 al 2010}{Membro della consulta provinciale degli studenti}
\ecvitem{Marzo 2010 e Marzo 2011}{Partecipazione alla gara di matematica a
	squadre "Coppa Aurea"}
\ecvitem{Ottobre 2010}{Eletto rappresentante di istituto. Nello stesso anno ha
	organizzato una giornata di incontro con l'astrofisica Margherita Hack
	che ha intervistato a nome di tutti gli studenti}
\ecvitem{Febbraio 2011}{Partecipazione alla Gara di Secondo Livello delle
	Olimpiadi Italiane della Fisica}
\ecvitem{Luglio 2011}{Maturità presso il Liceo Scientifico "Duca degli Abruzzi"
	di Gorizia}
\ecvitem{29 Settembre 2014}{Laurea triennale in Ingegneria dell'Informazione
	presso l'Università degli studi di Padova (voto di laurea: 110/110)}
\ecvitem{}{Titolo della tesi: \emph{Studio dell'efficienza di oscillatori LC
	integrati per impulsatori Ultra-Wideband}}
\ecvitem{12 Settembre 2016}{Laurea magistrale con lode in Ingegneria
	Elettronica presso l'Università degli studi di Padova}
\ecvitem{}{Titolo della tesi: \emph{Progetto di un amplificatore di potenza
	integrato in classe J in tecnologia bipolare}}
\ecvitem{da Ottobre 2016}{Ph.D. presso l'Univeristà degli studi di Padova nell'
	ambito di circuti integrati per applicazioni in radiofrequenza}

\ecvsection{Pubblicazioni}
\ecvitem{ESSCIRC 2017}{P. Scaramuzza, C. Rubino, M. Tiebout, M. Caruso,
	M. Ortner, A. Neviani, and A. Bevilacqua\\&
	\emph{Class-AB and Class-J 22dBm SiGe HBT PAs for X-Band Radar
	Systems}}

\clearpage
\ecvsection{Competenze personali}
\ecvmothertongue{Italiano}
\ecvlanguageheader{(*)}
\ecvlanguage{Inglese}{\ecvCOne}{\ecvCOne}{\ecvCOne}{\ecvCOne}{\ecvCOne}
\ecvlanguage{Tedesco}{\ecvBOne}{\ecvBOne}{\ecvATwo}{\ecvATwo}{\ecvATwo}
\ecvlanguagefooter{(*)}
\ecvitem{}{Conoscenza della lingua tedesca affinata tramite soggiorni all'estero
	durante il liceo:}
\ecvitem{Luglio 2008}{Corso presso la scuola Euroinstitute(Insbruck, Austria)}
\ecvitem{Luglio 2009}{Corso presso la scuola ActiLingua Academy (Vienna, Austria)}
\ecvitem{Agosto 2010}{Corso presso la scuola Institut f\"ur Interkulturelle
	Komunikation (Jena, Germania)}
\ecvitem{Marzo 2011}{Certificazione di lingua inglese di livello C1 (CAE)}
\ecvitem{Competenze informatiche}{Conoscenza di vari linguaggi di
	programmazione e scripting tra cui: C/C++, Java, Javascript, HTML, CSS,
	PHP, MATLAB e linguaggi funzionali}
\ecvitem{}{Conoscenze di database (SQL), networking e abilità da sistemista con
	sistemi operativi Linux e BSD}
\ecvitem{Elettronica}{Da sempre interessato all'elettronica con parecchi
	progetti all'attivo, in particolare in ambito analogico.\\&
	Conoscenza del linguaggio VHDL.}
\ecvitem{Profilo github}{\url{https://github.com/electricant}}

\ecvsection{Attività}
\ecvitem{da Febbraio 2014}{Segretario e socio attivo presso l'Associazione di
	promozione sociale Faber Libertatis di Padova, associazione impegnata
	nell'ambito del recupero di materiale informatico obsoleto e nel suo
	ripristino con Software Libero per scopi di utilità sociale.}
\ecvitem{Giugno 2015}{Vincitore della borsa di studio pre-laurea Roberto Rocca.}
\ecvitem{da Settembre 2015}{Membro del progetto Morpheus dell'Università di
	Padova dove si occupa della sottosezione elettronica. Il progetto	consiste
	nella realizzazione di un prototipo di rover marziano per competere nella
	\emph{European Rover Challenge} che si tiene in Polonia.}
\ecvitem{Ottobre-Dicembre 2015}{Docente del corso di alfabetizzazione
	informatica presso la Biblioteca Civica di Saonara (PD), organizzato con
	la collaborazione del Comune di Saonara e dell'APS Faber Libertatis.}
\ecvitem{a.a. 2015-2016}{Attività di tutorato presso il centro formativo
	maschile ONAOSI di Padova}
\ecvitem{dal 2014 al 2016}{Attività di supporto allo studio per studenti delle
	scuole superiori e universitari}
%\ecvsection{Allegati}
%\ecvitem{}{Autocertificazione laurea triennale con esami}
%\ecvitem{}{Autocertificazione iscrizione laurea magistrale con esami}
\end{europecv}
\end{document}
